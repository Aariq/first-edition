% !TEX TS-program = pdflatex
% The line above tells TeXShop to use the latex -> dvi -> distiller 
% path to produce the pdf. 

\documentclass[12pt]{article}

\usepackage{amsmath,amssymb,amsfonts}
\usepackage{courier,type1cm,array}
\usepackage{makeidx,graphicx,multicol}
\usepackage[bottom]{footmisc} % places footnotes at page bottom
\usepackage{natbib,float}
\usepackage[vmargin=1in,hmargin=1in]{geometry}

% allows compilation with pdflatex, under Windows 
\usepackage{epstopdf}

% Coloring of R code listings
\usepackage[formats]{listings}
\usepackage{color}
\definecolor{mygreen}{rgb}{0.1,0.5,0.1}
\definecolor{mygray}{rgb}{0.5,0.5,0.5}
\definecolor{mymauve}{rgb}{0.58,0,0.82}
\definecolor{mygrey}{rgb}{0.3,0.3,0.1}
\lstset{
language=R,
otherkeywords={data.frame},
basicstyle=\normalsize\ttfamily, 
commentstyle=\normalsize\ttfamily,
keywordstyle=\normalsize\ttfamily,
stringstyle=\color{mymauve}, 
commentstyle=\color{mygreen},
keywordstyle=\color{blue},
showstringspaces=false, xleftmargin=2.5ex,
columns=flexible,
literate={~}{{$\sim \; \; $}}1,
alsodigit={\.,\_},
deletekeywords={on,by,data,R,Q,mean,var,sd,log,family,na,options,q,weights,effects,matrix,nrow,ncol,wt,fix,distance},
}
\lstset{escapeinside={(*}{*)}} 

\lstdefineformat{Rpretty}{
	; = \space,
	\, = [\ \,\]]\string\space,
	<- = [\ ]\space\string\space,
	\= = [\ ]\space\string\space}

\usepackage[compact]{titlesec} 
\usepackage{enumitem}
\setlist{leftmargin=0.75cm}

\usepackage{array}
\newcolumntype{L}[1]{>{\raggedright\let\newline\\\arraybackslash\hspace{0pt}}m{#1}}
\newcolumntype{C}[1]{>{\centering\let\newline\\\arraybackslash\hspace{0pt}}m{#1}}
\newcolumntype{R}[1]{>{\raggedleft\let\newline\\\arraybackslash\hspace{0pt}}m{#1}}

% reduce white space above and below verbatim text blocks
\usepackage{etoolbox}
\makeatletter
\preto{\@verbatim}{\topsep=0.5pt \partopsep=0.5pt }
\makeatother

% let figs,boxes etc. occupy most of a page near where they
% appear in the source file, instead of getting pushed to
% the chapter's end. 
\renewcommand{\topfraction}{0.9}	
\renewcommand{\bottomfraction}{0.9}	
\renewcommand{\textfraction}{0.07}	
\renewcommand{\floatpagefraction}{0.85}	
% floatpagefraction MUST be less than topfraction
\setcounter{topnumber}{2}
\setcounter{bottomnumber}{2}
\setcounter{totalnumber}{4} 	

% Define Box environment for numbered boxes. 
\newcounter{box}
\newcommand{\boxnumber}{\addtocounter{box}{1} \thebox \thinspace}

\floatstyle{boxed}
\newfloat{Box}{tcph}{box}[section]
\numberwithin{Box}{section}

% A few shortcuts
% Define \z so that \z_0 and z'_0 have the same kerning for _0. 
\def\z{z^{}}
\def\Z{\mathbf{Z}}
\def\N{\mathbf{N}}

\newcommand{\be}{\begin{equation}}
\newcommand{\ee}{\end{equation}}
\newcommand{\ba}{\begin{equation} \begin{aligned}}
\newcommand{\ea}{\end{aligned} \end{equation}}

\makeindex

\usepackage[scaled=0.9]{zi4}
\usepackage{textcomp}


\linespread{1.2}

%%%%%%%%%%%%%%%%%%%%%%%%%%%%%%%%%%%%%%%%%%%%%%%%%%%%%%%%%%%%%%%%%%%%%
%%%%% End of the preamble 
 
\begin{document}

\author{Stephen P. Ellner, Dylan Z. Childs and Mark Rees}
\title{Errata for: \\ \emph{Data-driven Modeling of Structured Populations: 
A Practical Guide to the Integral Projection Model}} 

\date{Last update: \today} 

\maketitle

\paragraph{Page 15:} Equation (2.3.6), $C_1$ should be $C_0$ as it is in the life-cycle diagram, Figure 2.2. 

\paragraph{Section 2.3:} We think that the statement about ``piecewise continuous'' in the footnote is true, but at least one proof in the book isn't valid regardless of what the curves are that divide $\Z^2$ into subregions. A slightly less general definition, which should be sufficient for any applications, is as follows. For the basic
model where the individual-level state space $\Z$ is a 
bounded interval $[L,U]$, a partition of $\Z$ is a set of 
intervals 
\begin{equation}
\Z_1 = (z_0,z_1), \Z_2 = (z_1,z_2), \cdots, \Z_m = (z_{m-1},z_m) 
\end{equation}
where 
\be
L=z_0 < z_1 < z_2 < \cdots < z_M=U. 
\ee
A partition breaks $\Z^2$ into a set of open rectangles 
$$\Z_{ij} = \Z_i \times \Z_j = \{(z',z): z' \in \Z_i, z \in \Z_j\}.$$
Define the kernels $K_{ij}$ to be $K$ restricted to $\Z_{ij}$. 

We say that $K$ is \emph{piecewise continuous} if there exists a 
partition such that each of the kernels $K_{ij}$ is continuous on $\Z_{ij}$, 
and can be defined on the boundary of $\Z_{ij}$ so that it is continuous
on the closed rectangle $\bar{\Z}_{ij}$ consisting of $\Z_{ij}$ and its boundary.  

The reason this definition works is the general theory in Chapter 6
(originally in the Appendices to \citet{ellner-rees-2006}) applied to the  
closed intervals $\bar{\Z}_{i} =  [z_0,z_1]$ as a set of continuous components, 
with continuous component-to-component kernels $K_{ij}$. There are then state distribution 
functions $n_i(z,t)$ defined on each $\bar{\Z}_i$, and in terms of the general     
theory, this is an IPM with continuous kernels. But we can also think of it as defining a 
single distribution function $n(z,t)$ on all of $[L,U]$, consisting of $n_1,n_2, \cdots n_M$ 
side-by-side. 

Each of the points $z_i$ is in two adjacent components, but this doesn't matter because  
single points contribute nothing to the 
As an example consider  $\Z = [0,2]$ and the (nonsensical) kernel $K(z',z) = 1$ 
if $z'>1$, and 0 otherwise. The partition is $\Z_1 = (0,1), \Z_2 = (1,2)$ and the kernels are 
\be
K_{11}=K_{12}\equiv 0, K_{21}=K_{22}\equiv 1.
\ee
The population dynamics are 
\ba
n_1(z',t+1) & = \int_0^1 K_{11}(z',z) n_1(z,t) dz + \int_1^2 K_{12}(z',z) n_2(z,t) dz  = 0 \\
n_2(z',t+1) & = \int_0^1 K_{21}(z',z) n_1(z,t) dz + \int_1^2 K_{22}(z',z) n_2(z,t) dz  \\
& =  \int_0^1 n_1(z,t) dz + \int_1^2 n_2(z,t) dz.  
\label{eqn:badK1}
\ea
So $n_1(1,t+1)=0$, while $n_2(1,t+1)>0$ unless there were no individuals at time $t$. 
However, the values of $n_1$ and $n_2$ at the one point $z=1$ have no effect on the
integrals in the population dynamics, so we can regard $n(1,t+1)$ as being undefined,
or give it an arbitrary value such the average $n_1(1,t+1)$ and $n_2(1,t+1)$. 

The contrived kernel is \eqref{eqn:badK1} a counter-example to the claim in section 6.9 that a 
piecewise continuous kernel, as defined in that section, maps $L_1(\Z)$ into $C(\Z)$. 
The gap in the proof is the assertion that the functions $f_n$ converge almost everywhere, 
which is not necessarily true regardless of how the partitioning into sets $\mathcal{U}_k$ is done. But on the 
set of domains $\bar{\Z}_{i}$ the component kernels are all continuous and the $f_n$ converge 
pointwise, which is sufficient for the rest of the proof. 
In example \eqref{eqn:badK1}, $n_1$ is continuous on $\bar{\Z}_1$ and 
$n_2$ is continuous on $\bar{\Z}_2$, and that is exactly what it means to be continuous
on the state space with domains $\bar{\Z}_1$ and $\bar{\Z}_2$.  


\end{document} 





